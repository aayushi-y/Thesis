\chapter{Discussions and conclusion}
The chapter tries to conclude the whole work. It will be presenting in short what is the conclusion drawn from the proposed method. Limitation of the proposed method and what work can be carried out in the future to address the limitation and make it more effective have been discussed.
\section{Contributions}
In the proposed research, the active management scheme to exploit the spatial correlation between the sensed data were investigated. The main objective was the development of a technique that improves performance and provide dynamic drop probability. 
\par
The proposed packet dropping technique was found to be effective in improving the packet loss and delay performance for dense deployment of WSN.    When the packets are queued in Drop tail and RED mechanism, the packet drop ratio significantly increases, as the number of nodes increases. The proposed algorithm selectively drops packets according to correlation using data comparison in order to effectively utilize the buffer resource as much as possible. The main motive behind dropping the redundant packets is to improve the congestion in the network. 
\par The performance metrics of interest were \textit{packet loss ratio, end-to-end delay and network lifetime}.It was found that proposed scheme minimizes the packet loss due to buffer overflow, by dropping redundant packets and decreases the average delay.
\section{Limitations and Future work}
The packet dropping algorithm presented needs to be further studied. In its current form, the proposed algorithm requires access to payload in the packet, which makes the implementation in an intermediate node quite complex. Ways to simplify it by embedding the packets at the source with predefined congestion states are to be explored.