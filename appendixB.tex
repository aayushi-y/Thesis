\chapter{Klienrock Independent Assumption}	\OnePageChapter         % ONE PAGE!
Consider a network of communication links as shown in Figure B.1. Assume that there are several packet streams, each following a unique path that consists of a sequence of links through the network. Let x\textsubscript{s}, in packets/sec, be the arrival rate of the packet stream s. Then the total arrival rate at link (i, j) is
\begin{equation}
    \lambda \subscript{(i,j)} = \sum x\textsubscript{s}
\end{equation}
\begin{figure}[h!]
    \centering
    \includegraphics{Thesis/figs/append.png}
    \caption{Network of communication links}
    \label{fig:my_label}
\end{figure}
Assume that no packets travel in a loop, let x\textsubscript{s} denote the arrival rate of packet
stream s, and let f\textsubscript{(i,j)}(s) denote the fraction of the packets of stream s that go through link (i,j). Then the total arrival rate at link (i, j) is :
\begin{equation}
    \lambda(i,j) = \sum f\subscript{(i,j)} x\textsubscript{s}
\end{equation}
We know from the special case of two tandem queues that even if the packet streams are
Poisson with independent packet lengths at their point of entry into the network, this
property is lost after the first transmission line. To resolve the dilemma, it was suggested
by Kleinrock that merging several packet streams on a transmission line has an effect
\par 
It was concluded that it is often appropriate to adopt an M/M/1 queueing model for each
communication link regardless of the interaction of traffic on this link with traffic on
other links. This is known as Kleinrock independence approximation.

